

\documentclass[
	% -- opções da classe memoir --
	12pt,				% tamanho da fonte
	openright,			% capítulos começam em pág ímpar (insere página vazia caso preciso)
	twoside,			% para impressão em recto e verso. Oposto a oneside
	a4paper,			% tamanho do papel. 
	% -- opções da classe abntex2 --
	%chapter=TITLE,		% títulos de capítulos convertidos em letras maiúsculas
	%section=TITLE,		% títulos de seções convertidos em letras maiúsculas
	%subsection=TITLE,	% títulos de subseções convertidos em letras maiúsculas
	%subsubsection=TITLE,% títulos de subsubseções convertidos em letras maiúsculas
	% -- opções do pacote babel --
	english,			% idioma adicional para hifenização
	french,				% idioma adicional para hifenização
	spanish,			% idioma adicional para hifenização
	brazil,				% o último idioma é o principal do documento
	]{abntex2}


% ---
% PACOTES
% ---

% ---
% Pacotes fundamentais 
% ---
\usepackage{lmodern}			% Usa a fonte Latin Modern
\usepackage[T1]{fontenc}		% Selecao de codigos de fonte.
\usepackage[utf8]{inputenc}		% Codificacao do documento (conversão automática dos acentos)
\usepackage{indentfirst}		% Indenta o primeiro parágrafo de cada seção.
\usepackage{color}				% Controle das cores
\usepackage{graphicx}			% Inclusão de gráficos
\usepackage{microtype} 			% para melhorias de justificação
% ---

% ---
% Pacotes adicionais, usados no anexo do modelo de folha de identificação
% ---
\usepackage{multicol}
\usepackage{multirow}
% ---
	
% ---
% Pacotes adicionais, usados apenas no âmbito do Modelo Canônico do abnteX2
% ---
\usepackage{lipsum}				% para geração de dummy text
% ---

% ---
% Pacotes de citações
% ---
\usepackage[brazilian,hyperpageref]{backref}	 % Paginas com as citações na bibl
\usepackage[alf]{abntex2cite}	% Citações padrão ABNT

% --- 
% CONFIGURAÇÕES DE PACOTES
% --- 

% ---
% Configurações do pacote backref
% Usado sem a opção hyperpageref de backref
\renewcommand{\backrefpagesname}{Citado na(s) página(s):~}
% Texto padrão antes do número das páginas
\renewcommand{\backref}{}
% Define os textos da citação
\renewcommand*{\backrefalt}[4]{
	\ifcase #1 %
		Nenhuma citação no texto.%
	\or
		Citado na página #2.%
	\else
		Citado #1 vezes nas páginas #2.%
	\fi}%
% ---

% ---
% Informações de dados para CAPA e FOLHA DE ROSTO
% ---
\titulo{Trabalho Prático Final -- GCC118  Programação Matemática -- PCC540  Linear and Integer Programming }
\autor{ Douglas Geovanini de Paiva Mosca Leite, \\ Luiz Otávio... \\ \& Marcos...}
\local{Brasil}
\data{2025, Lavras}
\instituicao{%
  Universidade Federal de Lavras -- UFLA
 }
\tipotrabalho{Relatório técnico}
% O preambulo deve conter o tipo do trabalho, o objetivo, 
% o nome da instituição e a área de concentração 
\preambulo{Trabalho Prático Final -- GCC118  Programação Matemática -- PCC540  Linear and Integer Programming\LaTeX.}
% ---

% ---
% Configurações de aparência do PDF final


% alterando o aspecto da cor azul
\definecolor{blue}{RGB}{41,5,195}

% informações do PDF
\makeatletter
\hypersetup{
     	%pagebackref=true,
		pdftitle={\@title}, 
		pdfauthor={\@author},
    	pdfsubject={\imprimirpreambulo},
	    pdfcreator={LaTeX with abnTeX2},
		pdfkeywords={abnt}{latex}{abntex}{abntex2}{relatório técnico}, 
		colorlinks=true,       		% false: boxed links; true: colored links
    	linkcolor=blue,          	% color of internal links
    	citecolor=blue,        		% color of links to bibliography
    	filecolor=magenta,      		% color of file links
		urlcolor=blue,
		bookmarksdepth=4
}
\makeatother
% --- 

% --- 
% Espaçamentos entre linhas e parágrafos 
% --- 

% O tamanho do parágrafo é dado por:
\setlength{\parindent}{1.3cm}

% Controle do espaçamento entre um parágrafo e outro:
\setlength{\parskip}{0.2cm}  % tente também \onelineskip

% ---
% compila o indice
% ---
\makeindex
% ---

% ----
% Início do documento
% ----
\begin{document}

% Seleciona o idioma do documento (conforme pacotes do babel)
%\selectlanguage{english}
\selectlanguage{brazil}

% Retira espaço extra obsoleto entre as frases.
\frenchspacing 

% ----------------------------------------------------------
% ELEMENTOS PRÉ-TEXTUAIS
% ----------------------------------------------------------
% \pretextual

% ---
% Capa
% ---
\imprimircapa
% ---

% ---
% Folha de rosto
% (o * indica que haverá a ficha bibliográfica)
% ---
\imprimirfolhaderosto*
% ---

% ---
% Anverso da folha de rosto:
% ---

{
\ABNTEXchapterfont
}

% ---
% RESUMO
% ---

% resumo na língua vernácula (obrigatório)
\setlength{\absparsep}{18pt} % ajusta o espaçamento dos parágrafos do resumo
\begin{resumo}
 O Problema de Balanceamento de Linhas de Produção e Designação de Trabalhadores (ALWABP) consiste em atribuir tarefas a estações de trabalho e designar trabalhadores a essas estações, considerando relações de precedência entre tarefas, tempos de execução variáveis por trabalhador, e restrições de incapacidade específicas, com o objetivo de minimizar o tempo de ciclo da linha de produção, definido como o maior tempo de execução entre todas as estações. [Escrever sobre a heuristica]

 \noindent
\end{resumo}
% ---

% ---
% inserir lista de ilustrações
% ---
\pdfbookmark[0]{\listfigurename}{lof}
\listoffigures*
% ---

% inserir lista de tabelas
% ---
\pdfbookmark[0]{\listtablename}{lot}
\listoftables*

% ---
% inserir lista de abreviaturas e siglas
% ---
\begin{siglas}
  \item[ABNT] Associação Brasileira de Normas Técnicas
  \item[abnTeX] ABsurdas Normas para TeX
\end{siglas}
% ---

% ---
% inserir lista de símbolos
% ---
\begin{simbolos}
  \item[$ \Gamma $] Letra grega Gama
  \item[$ \Lambda $] Lambda
  \item[$ \zeta $] Letra grega minúscula zeta
  \item[$ \in $] Pertence
\end{simbolos}
% ---

% ---
% inserir o sumario
% ---
\pdfbookmark[0]{\contentsname}{toc}
\tableofcontents*
\cleardoublepage
% ---


% ----------------------------------------------------------
% ELEMENTOS TEXTUAIS
% ----------------------------------------------------------
\textual

% ----------------------------------------------------------
% Introdução (exemplo de capítulo sem numeração, mas presente no Sumário)
% ----------------------------------------------------------
\chapter{O problema}

O \textbf{ALWABP} (Assembly Line Worker Assignment and Balancing Problem) é um problema de otimização que combina balanceamento de linha de produção com designação de trabalhadores. Suas características principais são:

\section{Características Fundamentais}
\begin{itemize}
    \item Linha de produção com $m$ estações ordenadas linearmente
    \item Conjunto de $k$ trabalhadores, onde $|S| = |W|$ (número de estações igual ao número de trabalhadores)
    \item Conjunto de $n$ tarefas a serem distribuídas pelas estações
    \item Relações de precedência entre tarefas definidas por um grafo direcionado $G = (V, E)$
    \item Tempos de execução variáveis: $t_{wi}$ representa o tempo da tarefa $i$ pelo trabalhador $w$
    \item Restrições de incapacidade: $I_w$ define tarefas que o trabalhador $w$ não pode executar
\end{itemize}

\section{Restrições do Problema}
\begin{itemize}
    \item Cada tarefa é designada a exatamente uma estação
    \item Cada trabalhador é alocado a exatamente uma estação
    \item Cada estação possui exatamente um trabalhador
    \item Precedências devem ser respeitadas: se $i \preceq j$, então $i$ em estação anterior ou igual a $j$
    \item Tarefas em $I_w$ não podem ser executadas pelo trabalhador $w$
\end{itemize}

\section{Função Objetivo}
Minimizar o \textbf{tempo de ciclo} da linha, definido como o maior tempo de execução entre todas as estações.

\subsection*{Informações Disponíveis}
Para a implementação, dispõe-se de:
\begin{itemize}
    \item Número de estações ($m$), trabalhadores ($k$) e tarefas ($n$)
    \item Matriz de tempos $t_{wi}$ para todas combinações trabalhador-tarefa
    \item Conjuntos $I_w$ de tarefas impossíveis para cada trabalhador
    \item Grafo de precedências entre as tarefas
\end{itemize}
\chapter{Formulação Matemática do Problema}

A seguir apresenta-se a formulação do problema de Balanceamento de Linhas de Produção com Designação de Trabalhadores (ALWABP) como um modelo de Programação Linear Inteira. O objetivo é minimizar o tempo de ciclo da linha, considerando a heterogeneidade dos trabalhadores, restrições operacionais e relações de precedência entre tarefas.

\subsection{Variáveis de Decisão}

\begin{itemize}
    \item $v_{sw} \in \{0,1\}$: variável binária que indica se o trabalhador $w$ é alocado à estação $s$.
    \item $z_{siw} \in \{0,1\}$: variável binária que indica se a tarefa $i$ é executada na estação $s$ pelo trabalhador $w$.
    \item $C \in \mathbb{R}_+$: variável contínua que representa o tempo de ciclo da linha, definido como o maior tempo de processamento entre as estações.
\end{itemize}

\subsection{Função Objetivo}

O objetivo consiste em minimizar o tempo de ciclo da linha de produção. O tempo de ciclo é linearizado impondo que, para cada estação, o tempo total de execução das tarefas alocadas ao trabalhador desta estação não exceda $C$.

\[
\min C
\]

\subsection{Restrições}

\subsubsection*{(1) Atribuição única de cada tarefa}

Cada tarefa deve ser executada exatamente uma vez, em uma única estação e por um único trabalhador:

\[
\sum_{s \in S}\sum_{w \in W} z_{siw} = 1 
\quad \forall i \in N
\]

\subsubsection*{(2) Um trabalhador por estação}

Cada estação deve possuir exatamente um trabalhador alocado:

\[
\sum_{w \in W} v_{sw} = 1 
\quad \forall s \in S
\]

\subsubsection*{(3) Trabalhador em apenas uma estação}

Cada trabalhador pode ocupar apenas uma estação da linha:

\[
\sum_{s \in S} v_{sw} = 1 
\quad \forall w \in W
\]

\subsubsection*{(4) Vinculação tarefa–estação–trabalhador}

Uma tarefa só pode ser atribuída a um trabalhador se este estiver de fato na estação selecionada:

\[
z_{siw} \le v_{sw}
\quad \forall s \in S,\; \forall i \in N,\; \forall w \in W
\]

\subsubsection*{(5) Incapacidades de execução}

Caso o trabalhador $w$ seja incapaz de executar a tarefa $i$ (ou seja, $i \in I_w$), tal combinação é proibida:

\[
z_{siw} = 0
\quad \forall s \in S,\; \forall w \in W,\; \forall i \in I_w
\]

\subsubsection*{(6) Definição linear do tempo de ciclo}

O tempo total de processamento das tarefas atribuídas à estação $s$ não pode exceder o tempo de ciclo $C$:

\[
\sum_{i \in N}\sum_{w \in W} t_{wi}\, z_{siw} \le C 
\quad \forall s \in S
\]

Essa restrição define o valor de $C$ como o maior carregamento entre as estações.

\subsubsection*{(7) Restrições de precedência}

Se a tarefa $i$ deve preceder a tarefa $j$ no processo produtivo, então $i$ deve ser alocada a uma estação numericamente menor ou igual àquela que executa $j$:

\[
\sum_{s \in S}\sum_{w \in W} s\, z_{siw}
\;\le\;
\sum_{s \in S}\sum_{w \in W} s\, z_{sjw}
\quad \forall (i,j) \in E
\]

\subsubsection*{(8) Domínio das variáveis}

\[
v_{sw}, z_{siw} \in \{0,1\}, 
\qquad 
C \ge 0
\]


% ----------------------------------------------------------
\bibliography{abntex2-modelo-references}

\end{document}
